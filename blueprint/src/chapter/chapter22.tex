\chapter{One square and an odd number of triangles}


\begin{definition}[valutaion on $\mathbb{R}$]
  \label{valuation}
  \uses{valuation_on_reals}
\end{definition}

\begin{definition}[Three-coloring of plane]
  \label{three_coloring}
  \uses{valuation}
  TODO
\end{definition}

\begin{definition}[Rainbow triangle]
  \label{rainbow_triangle}
  \uses{three_coloring}
  TODO
\end{definition}


\begin{lemma}
  \label{ch22lemma1}
  For any blue point $p_0 = (x_b, y_b)$, green point $(x_g, y_g)$, and
  red point $(x_r, y_r)$, the $v$-value of the determinant
  \[
  \det \begin{bmatrix}
    x_b & y_b & 1 \\
    x_g & y_g & 1 \\
    x_r & y_r & 1
  \end{bmatrix}
  \] is at least $1$.
\end{lemma}
\begin{proof}
  TODO
\end{proof}

\begin{corollary}
  \label{ch22corollary}
  Any line of the plane receives at most two different colors.
  The area of a rainbow triangle cannot be $0$, and it cannot be
  $\frac{1}{n}$ for odd $n$.
  \uses{three_coloring, rainbow_triangle}
\end{corollary}
\begin{proof}
  \uses{ch22lemma1}
  Follow from \ref{ch22lemma1}
\end{proof}

\begin{lemma}
  \label{ch22lemma2}
  Every dissection of the unit square $S = [0, 1]^2$ into finitely
  many triangles contains an odd number of rainbow triangles, and thus at least one.
\end{lemma}
\begin{proof}
  TODO
\end{proof}


\begin{theorem}[Monsky's theorem]
  \label{monsky_theorem}
  It is not possible to dissect a square into an odd number of triangles of equal algebra area.
\end{theorem}
\begin{proof}
  \uses{ch22lemma2, ch22corollary}
  TODO
\end{proof}


\section*{Appendix: Extending valuations}

\begin{lemma}
  \label{valuation_lemma}
  A proper subring $R\subset K$ is a valuation ring with respect to some
  valuation $v$ into some ordered group $G$ if and only if $K = R \cup R^{-1}$.
\end{lemma}
\begin{proof}
  TODO
\end{proof}

\begin{theorem}
  \label{valuation_on_reals}
  The field of real numbers $\mathbb{R}$ has a non-Archimedean valuation
  to an ordered abelian group
  \[v: \mathbb{R} \to \{0\} \cup G\]
  such that $v(\frac{1}{2}) > 1$.
\end{theorem}
\begin{proof}
  \uses{valuation_lemma}
  TODO
\end{proof}


\section{Corollary}

The subgoal that we want to show is that whenever we have a dissection of the unit square into
triangles the sum of the areas of these triangles is given equal to $1$. Although this statement
may just look like a direct application of the additivity of the Lebesgue measure, there are
some issues. Firstly the definition of area we use in the proof, is not definitionally equal
to the actual Lebesgue measure of the triangle. The second is that the concept of a triangle is
not uniquely defined in $\mathbb{R}^2$: Is it an (ordered) set of three points? In that case the
triangle does not really have a sensible area. Is it the open hull of these three points, or is it
the closed hull of these three points? For the latter cases, the Lebesgue measure is the same, but
this does not follow immediately. The third issue is that in order to use additivity of the Lebesgue
measure, sets have to be measurable. Although all the sets we have mentioned are measurable
(as a matter of fact, essentially any set you can think of is measurable), if want to use this, we
have to show this.

Of course, all of these problems can be fixed, but the question is how to this in the most efficient
way using the present tools in mathlib. The most straightforward way to calculate the area of the
triangle, would be to use the fact that the area under a curve is the integral of the function
describing this curve. Mathlib has this in the library, but it only appears to have the `open' version.
This method works nicely to calculate the area of the `standard triangle' (with points $(0,0)(1,0),(0,1))$,
but for more complicated triangles it may be a bit tricky, and for degenerate open triangles
(which are not actually open) this is probably not going to work. Therefore we can try to use a
change of variables formula: areas under the Lebesgue measure are translation invariant, and under
a linear transformation it should scale with the absolute value of the determinant (I haven't found
the appropriate theorem in Mathlib yet, but I am fairly confident it should be there).
A bit of a non-serious problem with this approach is that I still have to find a notation in Lean,
that matches with these statement.

In order to also show that the area of the open and closed triangle are the same, I would suggest
the following. As there does not appear to be a present theorem similar to the one we used for
the open triangles, we show that the closed triangle is equal to the open triangle united with the
`boundary'. I say `boundary', because I do not mean the topological boundary, but the three
line segments given by the endpoints of the triangle. It should be fairly easy to show that this
union indeed gives the closed triangle. Furthermore, almost by definition, the line segment
corrsesponding to $[0,1]x[0,0]$ has measure zero by definition. Then we can do a similar transformation
argument as we did for the triangles.

Now, for all of this to work, we need that all of the sets we talked about are measurable (maybe not
all, but at least most of these). Mathlib tells us that the (open) area under a curve is measurable,
and so the open standard triangle is measurable. Unfortunately, images of measurable sets do not
have to be measurable. We do however know that the pre image of an open set is measurable. This
would suggest that if a linear transformation is a bijection, (translations are always bijections)
that the corresponding transformed triangle is also measurable. For a degenerate linear map, we can
probably use the fact the sets that have measure zero are always Lebesgue measurable. The line segments
will be measurable for the same reason.

If all that is done, we can do the following, where $det(A)$ means the determinant version of the
area of the triangle $A$, $\mathring{A}$ means the open hull of $A$ (and so this set is not necessarily
open), $\delta A$ means the boundary as given by the closed line segments of the points of the triangle
and $\overline A$ means the closed hull of the triangle:
\begin{equation}
\begin{split}
  \sum_{\text{triangles} A} \det(A) & = \sum_{\text{triangles} A} \mu (\mathring {A}) \\
  & =\mu (\bigcup_{\text{triangles} A} \mathring{A}) \\
  & = \mu ((\bigcup_{\text{triangles} A} \mathring{A}) \cup (\bigcup_{\text{triangles} A} \delta A)) \\
  & = \mu (\bigcup_{\text{triangles} A} \mathring {A} \cup \delta A) \\
  & = \mu (\bigcup_{\text{triangles} A} \overline{A}) \\
  & = \mu ([0,1]^2) = 1


\end{split}
\end{equation}

The first step is that the determinant area actually corresponds to the actual area of the open
triangle, the second step is by the fact that all of these open triangles are measurable. The third
step is by that the boundaries and therefore the union of the boundaries has measure zero (and maybe
also by the fact they are measurable? Not sure actually). The remaining steps are just standard set theory,
except for the last which is by definition of the Lebesgue measure, except that we have to show
our definition of the unit square matches the Lebesgue measure definition of the unit square.
